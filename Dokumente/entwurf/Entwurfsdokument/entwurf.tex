\documentclass[parskip=full,11pt]{scrartcl}
\usepackage[utf8]{inputenc}

%\title{Simulator für wiederholte Spiele}
%\author{Sebastian Feurer, Peter Koepernik, Luc Mercatoris,\\Christian Schorr, Pierre Toussing}

% section numbers in margins:
\renewcommand\sectionlinesformat[4]{\makebox[0pt][r]{#3}#4}

% header & footer
\usepackage{scrlayer-scrpage}
\lofoot{\today}
\refoot{\today}
\pagestyle{scrheadings}

\usepackage[sfdefault,light]{roboto}
\usepackage[T1]{fontenc}
\usepackage[german]{babel}
\usepackage[yyyymmdd]{datetime} % must be after babel
\renewcommand{\dateseparator}{-} % ISO8601 date format
\usepackage{hyperref}
\usepackage{bbm}
\usepackage{amsmath} % for $\text{}$
\usepackage{amssymb}
\usepackage[nameinlink]{cleveref}
\crefname{figure}{Abb}{Abb}
\usepackage[section]{placeins}
\usepackage{xcolor}
\usepackage{graphicx}
\usepackage{subfig}
\usepackage{float} % für Fließumgebungen; Platzierung H verschiebt nicht
\usepackage{multirow}
\restylefloat{figure}
\hypersetup{
	pdftitle={Pflichtenheft},
	bookmarks=true,
}
\usepackage{csquotes}

\newcommand\urlpart[2]{$\underbrace{\text{\texttt{#1}}}_{\text{#2}}$}

\usepackage{pflichtenheft}

\usepackage[nonumberlist]{glossaries}

\usepackage[T1]{fontenc}
\usepackage[scaled=0.85]{beramono}

\begin{document}
\section{Einleitung}
TODO

\section{Pakete und Klassen}
\subsection{Paket \texttt{edu.kit.loop.model}}
Das Modell beinhaltet Klassen und Methoden zum Starten und Abbrechen von Simulationen, sowie zum Erstellen und Speichern von Konfigurationen, Stufenspielen, Strategien und Populationen.

\subsubsection{Class \texttt{UserConfiguration implements java.io.Serializable}}
//vielleicht im falschen Package

Diese Klasse repräsentiert eine vom Nutzer erstellte Konfiguration. Sie bietet Methoden zum Lesen aller zugehörigen Parameter.

Konstruktoren:

Methoden:
\begin{itemize}\itemsep -10pt
	\item \texttt{String getGameName()}
	\item[] Gibt den Namen des Stufenspiels dieser Konfiguration zurück.
	\item[] Returns: den Namen des Stufenspiels dieser Konfiguration
	
	\item \texttt{int getAgentCount()}
	\item[] Gibt die Anzahl von Agenten in dieser Konfiguration zurück.
	\item[] Returns: die Anzahl von Agenten in dieser Konfiguration
	
	\item \texttt{int getRoundCount()}
	\item[] Gibt die Anzahl von Runden pro Adaptionsschritt in dieser Konfiguration zurück.
	\item[] Returns: die Anzahl von Runden pro Adaptionsschritt in dieser Konfiguration
	
	\item \texttt{int getIterationCount()}
	\item[] Gibt die Anzahl der durchzuführenden Wiederholungen in dieser Konfiguration zurück.
	\item[] Returns: die Anzahl von Wiederholungen in dieser Konfiguration
	
	\item \texttt{List<String> getAvailableStrategyNames()}
	\item[] Gibt eine Liste der Namen von Strategien, die in der Konfiguration zugelassen sind zurück.
	\item[] Returns: eine Liste der Strategienamen
	
	\item \texttt{boolean getMixedAllowed()}
	\item[] Gibt zurück, ob in dieser Konfiguration gemischte Strategien zugelassen sind.
	\item[] Returns: \texttt{true}, falls in dieser Konfiguration gemischte Strategien zugelassen sind; \texttt{false} sonst
	
	\item \texttt{List<String> getPopulationNames()}
	\item[] Gibt eine Liste der Namen von Popultaionen, die in der Konfiguration ausgewählt sind zurück.
	\item[] Returns: eine Liste der Populationsnamen
	
	\item \texttt{String getPairBuilderName()}
	\item[] Gibt den Namen des  Paarbildungsalgorithmus dieser Konfiguration zurück.
	\item[] Returns: den Namen des Paarbildungsalgorithmus dieser Konfiguration
	
	\item \texttt{String getSuccessQuantifierName()}
	\item[] Gibt den Namen der Erfolgsquantifizierung dieser Konfiguration zurück.
	\item[] Returns: der Name der Erfolgsquantifizierung dieser Konfiguration
	
	\item \texttt{String getStrategyAdjusterName()}
	\item[] Gibt den Namen des Adaptionsmechanismus dieser Konfiguration zurück.
	\item[] Returns: den Namen des Adaptionsmechanismus dieser Konfiguration
	
	\item \texttt{List<double> getStrategyAdjusterParameters()}
	\item[] Gibt eine Liste der Werte der Parameter des \texttt{StrategyAdjuster}s zurück
	\item[] Returns: die Liste der Werte

	
	\item \texttt{String getEquilibriumCriterionName()}
	\item[] Gibt den Namen des  Gleichgewichtskriterium dieser Konfiguration zurück.
	\item[] Returns: den Namen des Gleichgewichtskriterium dieser Konfiguration
	
	\item \texttt{List<double> getEquilibriumParameters()}
	\item[] Gibt eine Liste der Werte der Parameter des \texttt{EquilibriumCriterion}s zurück
	\item[] Returns: die List der Werte
	
	\item \texttt{int getMaxAdapts()}
	\item[] Gibt die maximale Zahl durchzuführender Adaptionsschritte für diese Konfiguration zurück.
	\item[] Returns: die maximale Zahl durchzuführender Adaptionsschritte für diese Konfiguration
	
	\item \texttt{String getVariableParameterName()}
	\item[] Gibt den Namen des variablen Parameters zurück.
	\item[] Returns: den Namen des variablen Parameters
	
	\item \texttt{double getStartValue()}
	\item[] Gibt den Startwert des variablen Parameters zurück
	\item[] Returns: den Startwert des variablen Parameters
	
	\item \texttt{double getEndValue()}
	\item[] Gibt den Endwert des variablen Parameters zurück
	\item[] Returns: den Endwert des variablen Parameters
	
	\item \texttt{double getStepSize()}
	\item[] Gibt eie Schrittweite der Werte des variablen Parameters zurück
	\item[] Returns: die Schrittweite der Werte des variablen Parameters
	
	
\end{itemize}

\subsubsection{Class \texttt{Configuration}}
Diese Klasse repräsentiert die elementare Konfiguration einer einzelnen Wiederholung und enthält alle Informationen zum Start einer solchen:
\begin{itemize}\itemsep -10pt
\item Stufenspiel
\item Anzahl von Agenten
\item Runden pro Wiederholung
\item Ob gemischte Strategien zugelassen sind
\item Gruppen-/Segmenteinteilungen
\item Kapital- und Strategieinitialisierung der Segmente
\item Agentenpaarung
\item Erfolgsquantifizierung
\item Adaptionsmechanismus
\item Gleichgewichtskriterium
\item Maximale Zahl von Adaptionsschritten
\end{itemize}

Konstruktoren:

Methoden:

\subsubsection{Inteface \texttt{Repository<T>}}

// Im falschen Paket

Ein Repository speichert eine Zuordnung von Namen und Entitäten und stellt diese auf Anfrage zur Verfügung.

Parameter:
\begin{itemize}\itemsep -10pt
	\item \texttt{<T>} Der Typ der Entitäten, die in dem Depot gespeichert werden sollen
\end{itemize}


Methoden:
\begin{itemize}\itemsep -10pt
	\item \texttt{T getEntityByName(String name)}
	\item[] Gibt die zu dem Namen gehörige Entität zurück oder \texttt{null} falls das Depot keine Entität unter dem gegebene Namen speichert
	\item[] \texttt{name}: der Name der Entität die abgefragt werden soll
	\item[] Returns: die zu dem Namen gehörige Entität oder \texttt{null}
	
	\item \texttt{boolean addEntity(String name, T entity)}
	\item[] Fügt dem Depot eine neue Entität unter dem gegebene Namen hinzu
	\item[] \texttt{name}: der Name der Entität, die hinzugefügt werden soll
	\item[] \texttt{entity}: Die Entität, die hinzugefügt werden soll
	\item[] Returns: \texttt{true}, wenn die Entität erfolgreich hinzugefügt wurde und \texttt{false}, wenn bereits eine Entität unter dem Name im Depot existiert 
	
	\item \texttt{boolean containsEntityName(String name)}
	\item[] Gibt zurück, ob das Depot eine Entität unter dem gegebenen Namen enthält
	\item[] \texttt{name}: der Name der geprüft werden soll
	\item[] Returns: \texttt{true}, wenn das Depot eine Entität unter diesem Namen speichert und \texttt{false}, wenn nicht
	
\end{itemize}

\subsubsection{Class \texttt{FileIO}}

Diese Klasse stellt Funktionalitäten zum Laden und Speichern von Entitäten in Dateien zur Verfügung.

Methoden:

\begin{itemize}\itemsep -10pt
	\item \underline{\texttt{static <T> T loadEntity(File file)}}
	\item[] Lädt die Entität des Typs \texttt{T} aus der angegebenen Datei 
	\item[] \texttt{<T>} Entitätentyp der geladen werden soll
	\item[] \texttt{file}: Die Datei aus der die Entität geladen werden soll
	\item[]Returns: die geladenen Entität oder \texttt{null} wenn keine Entität des Typs \texttt{T} geladen werden konnte
	
	\item \underline{\texttt{static <T> List<T> loadAllEntities(File dir)}}
	\item[] Lädt alle Entität des Typs \texttt{T} aus den Dateien in dem gegebene Ordner 
	\item[] \texttt{<T>} Entitätentyp der geladen werden soll
	\item[] \texttt{dir}: Der Ordner aus dem die Entität geladen werden soll
	\item[]Returns: eine Liste aller geladener Entitäten des Typs \texttt{T}
	
	\item \underline{\texttt{static <T> List<T> loadAllEntities(List<File> files)}}
	\item[] Lädt alle Entität des Typs \texttt{T} aus den gegebenen Dateien
	\item[] \texttt{<T>} Entitätentyp der geladen werden soll
	\item[] \texttt{files}: Die Liste der Dateien aus denen die Entitäten geladen werden sollen
	\item[]Returns: eine Liste aller geladener Entitäten des Typs \texttt{T}

	\item \underline{\texttt{static <T> boolean saveEntity(File file, T entity)}}
\item[] Speichert die Entität des Typs \texttt{T} in der gegebenen Datei
\item[] \texttt{<T>} Entitätentyp der gespeichert werden soll
\item[] \texttt{file}: Die Datei, in welche die Entität gespeichert werden soll
\item[] \texttt{entity}: Die Entität die gespeichert werden soll
\item[]Returns: \texttt{true}, wenn das Speichern erfolgreich war, andernfalls \texttt{false}
\end{itemize}



\subsection{Paket \texttt{edu.kit.loop.model.simulator}}
Dieses Paket enthält das Interface \texttt{Simulator}. Dieses bietet eine Schnittstelle zum Starten und Abbrechen von Simulationen. Beim Start einer Simulation wird eine Referenz auf ein \texttt{Simulation}-Objekt zurückgegeben, über das der Ausführungsstatus und die Ergebnisse der Simulation abgefragt werden können.

\subsubsection{Class \texttt{Simulation}}
Ein \texttt{Simulation}-Objekt enthält Informationen zu einer gestarteten Simulation, etwa deren Konfiguration, Ausführungsstatus, \texttt{id} und gegebenenfalls die Ergebnisse der Simulation. Es wird von einem \texttt{Simulator} erzeugt und bereitgestellt, wenn eine Simulation gestartet wird.

Die Klasse unterscheidet nicht zwischen Multikonfigurationen und Nicht-Multikonfigurationen, geht also allgemein von mehreren zugrundeliegenden elementaren Konfigurationen aus.

Konstruktoren:
\begin{itemize} \itemsep -10pt
\item \texttt{Simulation(UserConfiguration config, int id)}
\item[] Erzeugt ein neues \texttt{Simulation}-Objekt mit der gegebenen \texttt{UserConfiguration} und der gegebenen \texttt{id}.
\item[] \texttt{config}: die Konfiguration der Simulation
\item[] \texttt{id}: die \texttt{id} dieser Simulation
\end{itemize}

Methoden:
\begin{itemize}\itemsep -10pt
\item \texttt{protected void addIterationResult(IterationResult result, int i)}
\item[] Fügt ein Wiederholungsergebnis zur \texttt{i}-ten elementaren Konfiguration dieser Simulation als \texttt{IterationResult} hinzu.
\item[] \texttt{result}: das Wiederholungsergebnis, das hinzugefügt werden soll
\item[] \texttt{i}: die elementare Konfiguration, zu der das Wiederholungsergebnis hinzugefügt werden soll

\item \texttt{void registerIterationFinished(Consumer<IterationResult> action)}
\item[] Registriert eine Aktion, die jedes mal ausgeführt wird, wenn eine Wiederholung dieser Simulation abgeschlossen wird. Der Aktion wird als Argument das Ergebnis der Wiederholung als \texttt{IterationResult} übergeben.
\item[] \texttt{action}: Die Aktion, die bei Abschluss jeder Wiederholung dieser Simulation ausgeführt werden soll

\item \texttt{List<IterationResult> getIterationResults(int i)}
\item[] Gibt eine Liste der Wiederholungsergebnisse der bisher abgeschlossenen Wiederholungen mit der \texttt{i}-ten zugrundeliegenden elementaren Konfiguration als \texttt{IterationResult}s zurück.
\item[] \texttt{i}: die elementare Konfiguration, zu der die Wiederholungsergebnisse zurückgegeben werden sollen
\item[] Returns: die Wiederholungsergebnisse der bisher abgeschlossenen Wiederholungen mit der \texttt{i}-ten zugrundeliegenden elementaren Konfiguration

\item \texttt{UserConfiguration getUserConfiguration()}
\item[] Gibt die \texttt{UserConfiguration} dieser Simulation zurück.
\item[] Returns: die \texttt{UserConfiguration} dieser Simulation

\item \texttt{int getConfigurationCount()}
\item[] Gibt die Zahl der dieser Simulation zugrundeliegenden elementaren Konfigurationen zurück. Im Falle einer Multikonfiguration also die Anzahl verschiedener Werte des Multikonfigurationsparameters, ansonsten \(1\).
\item[] Returns: die Zahl der dieser Simulation zugrundeliegenden elementaren Konfigurationen

\item \texttt{int getId()}
\item[] Gibt die \texttt{id} dieser Simulation zurück.
\item[] Returns: die \texttt{id} dieser Simulation
\end{itemize}

\subsubsection{Interface \texttt{Simulator}}
Über einen Simulator können Simulationen gestartet und abgebrochen werden. Zum Starten einer Simulation muss dem \texttt{Simulator} eine \texttt{UserConfiguration} übergeben werden, die die durchzuführende Simulation spezifiziert. Daraufhin wird ein \texttt{Simulation}-Objekt erzeugt und zurückgegeben, über das der Ausführungsstatus und die Ergebnisse der gestarteten Simulation abgefragt werden können. Jeder Simulation wird beim Start eine eindeutige \texttt{id} zugewiesen.

Methoden:
\begin{itemize} \itemsep -10pt
\item \texttt{Simulation startSimulation(UserConfiguration config)}
\item[] Startet eine Simulation mit der gegebenen \texttt{UserConfiguration} und gibt ein \texttt{Simulation}-Objekt zu der gestarteten Simulation zurück.
\item[]\texttt{config}: die Konfiguration, mit der die Simulation ausgeführt werden soll
\item[] Returns: ein \texttt{Simulation}-Objekt zu der gestarteten Simulation

\item \texttt{Simulation startSimulation(UserConfiguration config, Consumer<Simulation> action)}
\item[] Startet eine Simulation mit der gegebenen Konfiguration und gibt ein \texttt{Simulation}-Objekt zu der gestarteten Simulation zurück. Führt die gegebene Aktion mit der zurückgegebenen \texttt{Simulation} als Argument aus, sobald die Ausführung der Simulation abgeschlossen ist.
\item[] \texttt{config}: die Konfiguration, mit der die Simulation ausgeführt werden soll
\item[] \texttt{action}: Die Aktion, die ausgeführt werden soll, sobald die Simulation abgeschlossen ist
\item[] Returns: ein \texttt{Simulation}-Objekt zu der gestarteten Simulation

\item \texttt{boolean stopSimulation(Simulation sim)}
\item[] Falls die gegebene Simulation aktuell läuft, wird diese abgebrochen und \texttt{true} zurückgegeben. Andernfalls wird \texttt{false} zurückgegeben.
\item[] \texttt{sim}: Die Simulation, die abgebrochen werden soll
\item[] Returns: \texttt{true}, wenn die Simulation erfolgreich abgebrochen wurde, \texttt{false} sonst

\item \texttt{boolean stopSimulation(int id)}
\item[] Falls eine Simulation mit der gegebenen \texttt{id} läuft, wird diese abgebrochen und \texttt{true} zurückgegeben. Andernfalls wird \texttt{false} zurückgegeben.
\item[] \texttt{id}: Die \texttt{id} der Simulation, die abgebrochen werden soll
\item[] Returns: \texttt{true}, wenn die Simulation erfolgreich abgebrochen wurde, \texttt{false} sonst

\item \texttt{void stopAllSimulation()}
\item[] Bricht die Ausführung aller gestarteten Simulationen ab.

\item \texttt{Simulation getSimulation(int id)}
\item[] Gibt das \texttt{Simulation}-Objekt der Simulation mit der entsprechenden \texttt{id} zurück, falls existent. Ansonsten \texttt{null}.
\item[] \texttt{id}: Die \texttt{id} der Simulation, deren \texttt{Simulation}-Objekt zurückgegeben werden soll
\item[] Returns: das \texttt{Simulation}-Objekt der Simulation mit der entsprechenden \texttt{id}, falls existent. Ansonsten \texttt{null}.
\end{itemize}

\subsubsection{Class \texttt{ThreadPoolSimulator}}
Implements: \texttt{Simulator}

Eine Implementierung des \texttt{Simulator}-Interfaces. Führt die Wiederholungen parallel in einem \texttt{ThreadPool} aus.

Konstruktoren:
\begin{itemize}\itemsep -10pt
\item \texttt{ThreadPoolSimulator()}
\item[] Erzeugt einen neuen \texttt{ThreadPoolSimulator}.

\item \texttt{ThreadPoolSimulator(int maxThreads)}
\item[] Erzeugt einen neuen \texttt{ThreadPoolSimulator} mit der gegebenen maximalen Anzahl von \texttt{Thread}s.
\item[] \texttt{maxThreads}: die maximale Anzahl von \texttt{Thread}s im \texttt{ThreadPool}
\end{itemize}

Methoden:
\begin{itemize}\itemsep -10pt
\item \texttt{int getRunningIterationCount()}
\item[] Gibt die Zahl aktuell ausgeführter Wiederholungen zurück.
\item[] Returns: die Zahl aktuell ausgeführter Wiederholungen

\item \texttt{int getQueuedIterationCount()}
\item[] Gibt die Zahl aktuell auf Ausführung wartender Wiederholungen zurück.
\item[] Returns: die Zahl aktuell auf Ausführung wartender Wiederholungen
\end{itemize}

\subsubsection{Class \texttt{ConfigurationCreator}}
Diese Klasse nimmt eine \texttt{UserConfiguration} entgegen und extrahiert daraus alle zugehörigen elementaren Konfigurationen. Diese werden als \texttt{Configuration}s zurückgegeben.

Konstruktoren:
\begin{itemize}\itemsep -10pt
\item \texttt{ConfigurationCreator()}
\item[] Erzeugt einen neuen \texttt{ConfigurationCreator}.
\end{itemize}

Methoden:
\begin{itemize}\itemsep -10pt
\item \texttt{List<Configuration> generateConfigurations(UserConfiguration config)}
\item[] Erzeugt alle zu der gegebenen \texttt{UserConfiguration} gehörigen elementaren Konfigurationen und gibt sie als \texttt{Configuration}s zurück.
\item[] \texttt{config}: Die \texttt{UserConfiguration}, zu der alle elementaren Konfigurationen generiert werden sollen
\item[] Returns: alle zu der gegebenen \texttt{UserConfiguration} gehörigen elementaren Konfigurationen als \texttt{Configuration}s
\end{itemize}

\subsection{Paket \texttt{edu.kit.loop.model.simulationengine}}

\subsubsection{Class \texttt{GameResult}}

Diese Klasse speichert das Ergebnis einer Runde zwischen zwei Agenten. Das Ergebnis besteht aus:
\begin{itemize}\itemsep -10pt
	\item Den beteiligten Agenten
	\item Den Aktionen der beteiligten Agenten
	\item Die erhaltenen Auszahlungen der Agenten
\end{itemize}
	
Konstruktor:

Methoden:
\begin{itemize}\itemsep -10pt
	\item \texttt{Agent getMainAgent()}
	\item[]Gibt den Hauptspieler der zum Ergebnis gehörenden Runde zurück.
	\item[]Returns: Der Hauptagent der Runde
	\item \texttt{Agent getOpponentAgent()}
	\item[]Gibt den Gegenspieler der zum Ergebnis gehörenden Runde zurück.
	\item[]Returns: Der Gegenspieler der Runde
	\item \texttt{int getMainAgentPayoff()}
	\item[] Gibt die in der zughörigen Runde erzielte Auszahlung des Hauptspielers zurück.
	\item[]Returns: Die erzielte Auszahlung
	\item \texttt{int getOpponentAgentPayoff()}
	\item[] Gibt die in der zughörigen Runde erzielte Auszahlung des Gegenspielers zurück.
	\item[]Returns: Die erzielte Auszahlung
	\item \texttt{boolean hasMainAgentCooperated()}
	\item[]Gibt zurück, ob der Hauptspieler in der zugehörigen Runde kooperiert hat oder nicht.
	\item[]Returns: 
	\item \texttt{boolean hasOpponentAgentCooperated()}
	\item[]Gibt zurück, ob der Gegenspieler in der zugehörigen Runde kooperiert hat oder nicht.
	\item[]Returns: 
\end{itemize}


\subsubsection{Interface \texttt{SimulationHistory}}
Eine Implementierung des \texttt{SimulationHistory}-Interfaces speichert für jeden Agenten die Ergebnisse aller bisherigen Runden eines Adaptionsschrittes. Es werden Methoden zum Hinzufügen ne bereitgestellt:

Methoden:
\begin{itemize}\itemsep -10pt
	\item \texttt{void addResult(GameResult result)}
	\item[] Fügt der Historie ein neues Rundenergebnis hinzu.
	\item[] \texttt{result}: Das \texttt{GameResult}, das zur Historie hinzugefügt werden soll
	\item \texttt{List<GameResult> getAllResults()}
	\item[]Gibt alle Rundenergebnisse  aller Agenten als Liste zurück.
	\item[]Returns: Die Liste der Rundenergebnisse
	\item \texttt{List<GameResult> getResultsByAgent(agent: Agent)}
	\item[]Gibt alle Rundenergebnisse des gegebenen Agenten als Liste zurück.
	\item[] \texttt{agent}: Der Agent dessen Ergebnisse abgefragt werden
	\item[]Returns: Die Liste der Rundenergebnisse
	\item \texttt{List<GameResult> getLatestResults()}
	\item[]Gibt das letzte Rundenergebnis aller Agenten als Liste zurück.
	\item[]Returns: Die Liste der Rundenergebnisse
	\item \texttt{GameResult getLatestResultByAgent(agent: Agent)}
	\item[] Gibt das letzte Rundenergebnis des gegebenen Agenten zurück.
	\item[] \texttt{agent}: Der Agent dessen letztes Ergebnis abgefragt wird
	\item[]Returns: Das letzte RundenErgebnis des Agenten als \texttt{GameResult}
	\item \texttt{Iterable<GameResult> getAllWhere(condition: Predicate<GameResult>)}
	\item[] Filtert alle Rundenergebnisse gemäß einer gegebenen Filterbedingung und liefert sie in Form eines \texttt{Iterable} zurück.
	\item[]\texttt{condition}: Die Filterbedinung als Prädikat mit einem \texttt{GameResult} als Parameter.
	\item[]Returns: Die gefilterten Rundenergebnisse als \texttt{Iterable}
	\item \texttt{GameResult getLatestWhere(condition: Predicate<GameResult>)}
	\item[] Filtert alle Rundenergebnisse gemäß einer gegebenen Filterbedingung und liefert das letzte Rundenergebnis, welches die Bedingung erfüllt zurück.
	\item[]\texttt{condition}: Die Filterbedinung als Prädikat mit einem \texttt{GameResult} als Parameter.
	\item[]Returns: Das letzte Rundenergebnis, welches die Filterbedingung erfüllt
\end{itemize}

\subsubsection{Class \texttt{SimulationHistoryTable}}
Implements: \texttt{SimulationHistory}

Diese Klasse implementiert das \texttt{SimulationHistory}-Interface.

\subsection{Paket \texttt{edu.kit.loop.model.plugin}}

\subsubsection{Class \texttt{PluginLoader}}

Diese Klasse stellt die Funktionalität zum Laden von Plugins mittels der Java \texttt{ServiceLoader}-API zur Verfügung

Methoden:

\begin{itemize}\itemsep -10pt
	\item \underline{\texttt{static <T> List<T> loadPlugins()}}
	\item[] Lädt alle verfügbaren Plugins des angegebenen Typs \texttt{T}
	\item[] \texttt{<T>} Plugintyp der geladen werden soll
	\item[]Returns: eine Liste der geladenen Plugins
\end{itemize}

\subsubsection{abstract Class \texttt{PluginControl extends javafx.scene.layout.Pane}}

Diese Klasse dient als Steuerelement für die Konfiguration eines Plugins. Sie stellt eine Funktionalität zum Abfragen der vom Benutzer eingegebenen Konfigurationsparameter zur Verfügung.

Methoden:

\begin{itemize}\itemsep -10pt
	\item \texttt{abstract List<double> getParameters()}
	\item[] Gibt eine Liste der eingegebene Parameter für die Pluginkonfiguration zurück
	\item[] Returns: eine Liste der eingegebenen \texttt{double}-Parameter
\end{itemize}

\subsubsection{Class \texttt{TextFieldPluginControl extends PluginControl}}

Diese Klasse ist eine Implementierung eines  \texttt{PluginControl}, bei der die Eingabe der Konfigurationsparameter über Textfelder erfolgt. Die Eingabe wird weiterhin bereits auf Korrektheit überprüft.

Konstruktoren:

\begin{itemize}\itemsep -10pt
	\item \texttt{TextFieldPluginControl(List<Parameter> params)}
	\item[] Erzeugt eine Instanz, die die übergebenen \texttt{Parameter} konfigurierbar macht.
	\item[] \texttt{params}: eine Liste der konfigurierbaren Parameter
\end{itemize}

Methoden:

\begin{itemize}\itemsep -10pt
	\item \texttt{void addParameter(Parameter param)}
	\item[] Fügt dem Steuerelement einen \texttt{Parameter} hinzu, der dann über das Steuerelement konfigurierbar ist
	\item[] \texttt{param}: Der Parameter, der zum Steuerelement hinzugefügt werden soll.
	\item \texttt{void addParameters(List<Parameter> params)}
	\item[] Fügt dem Steuerelement eine Liste von \texttt{Parameter}n hinzu, die dann über das Steuerelement konfigurierbar sind
	\item[] \texttt{params}: Die Parameter, die zum Steuerelement hinzugefügt werden sollen.
\end{itemize}

\subsubsection{Interface \texttt{PluginRenderer}}

Dieses Interface stellt die Funktionalität zum Erzeugen eines Konfigurations-Steuerelemet für Plugins in einer JavaFX Anwendung bereit.

Methoden:

\begin{itemize}\itemsep -10pt
	\item \texttt{PluginControl render()}
	\item[] Liefert eine \texttt{PluginControl}-Instanz zurück, die in ein JavaFX-Fenster eingefügt werden kann
\end{itemize}

\subsubsection{Class \texttt{GenericRenderer implements PluginRenderer}}

Diese Klasse implementiert das \texttt{PluginRenderer}-Interface und erzeugt für ein beliebiges Plugin ein generisches, Textfeld basiertes Konfigurations-Steuerelement (\texttt{TextFieldPluginControl}-Instanz).

Konstruktoren:

\begin{itemize}\itemsep -10pt
	\item \texttt{GenericRenderer(plugin: Plugin)}
	\item[] Erzeugt eine Instanz, die für das gegebene Plugin ein Konfigurations-Steuerelement erstellt
	\item[] \texttt{plugin}: Das Plugin welches der Renderer darstellen soll
	\item \texttt{GenericRednerer(params: List<Parameter>)}
	\item[] Erzeugt eine Instanz, die ein Steuerelement mit den gegebenen \texttt{Parameter} erzeugt.
	\item[] \texttt{params}: Eine Liste der konfigurierbaren Parameter eines Plugins
\end{itemize} 

\subsubsection{Class \texttt{Parameter implements Nameable}}

Diese Klasse definiert einen Konfigurationsparameter für Plugins. Sie bietet Funktionaliäten zum Festlegen des Wertebereichs und der Granularität des Konfigurationsparameter.

Konstruktoren:
\begin{itemize}\itemsep -10pt
\item \texttt{Parameter()}
\item[] Erzeugt eine neue Instanz mit unbeschränktem Wertebereich 
\item \texttt{Parameter(double minVal)}
\item[] Erzeugt eine neue Instanz mit nach unten beschränktem Wertebereich
\item[] \texttt{minVal}: untere Schranke des Wertebereichs
\item \texttt{Parameter(double maxVal)}
\item[] Erzeugt eine neue Instanz mit nach oben beschränktem Wertebereich
\item[] \texttt{maxVal}: obere Schranke des Wertebereichs
\item \texttt{Parameter(double minVal, double maxVal)}
\item[] Erzeugt eine neue Instanz mit nach unten und oben beschränktem Wertebereich
\item[] \texttt{minVal}: untere Schranke des Wertebereichs
\item[] \texttt{maxVal}: obere Schranke des Wertebereichs
\item \texttt{Parameter(double minVal, double maxVal, double stepSize)}
\item[] Erzeugt eine neue Instanz mit nach oben und unten beschränktem Wertebereich, sowie eingeschränkter Granularität
\item[] \texttt{minVal}: untere Schranke des Wertebereichs
\item[] \texttt{maxVal}: obere Schranke des Wertebereichs
\item[] \texttt{stepSize}: Granularität des Parameters

\end{itemize}

Methoden:

\begin{itemize}\itemsep -10pt
\item \texttt{double getMinValue()}
\item[] Gibt die untere Schranke des gültigen Wertebereichs zurück
\item[] Returns: die untere Schranke des gültigen Wertebereichs
\item \texttt{double getMaxValue()}
\item[] Gibt die obere Schranke des gültigen Wertebereichs zurück
\item[]Returns: die obere Schranke des gültigen Wertebereichs
\item \texttt{double getStepSize()}
\item[] Gibt die Granularität des Wertebereichs zurück
\item[]Returns: die Granularität des Wertebereichs
\end{itemize}

\subsubsection{Class \texttt{ParameterValidator}}

Diese Klasse bietet Funktionalitäten zum Validieren einer Parameterbelegung in Form von statischen Hilfsmethoden.

Methoden:

\begin{itemize} \itemsep -10pt
\item \underline{\texttt{static boolean isValueValid(double val, Parameter param)}}
\item[] Prüft, ob der gegeben Wert im Wertebereich des gegebenen Parameters liegt
\item[] \texttt{val}: der Wert der überprüft werden soll
\item[] \texttt{param}: der Parameter, dessen Wertebereich für die Überprüfung genutzt werden soll.
\item[] Returns: \texttt{true}, wenn der Wert im Wertebereich des Parameters liegt \texttt{false} andernfalls
\item \underline{\texttt{static double getClosestValid(double val, Parameter param)}}
\item[] Gibt den Wert im Wertebereich des Parameters zurück, der am nächsten an dem gegebenen Wert liegt
\item[] \texttt{val}: der Wert, der beim ermitteln betrachtet wird
\item[] \texttt{param}: der Parameter, desssen Wertebereich betrachtet wird
\item[] Returns: der Wert aus dem Wertebereich des Parameters, der dem gegebenen Wert am nächsten liegt
\end{itemize}

\subsubsection{abstract Class \texttt{Plugin<T> implements  Nameable}}

Diese Klasse stellt ein dynamisch ladbares Plugin da. Sie ist ein generischer Container für eine Klasse, die die eigentliche Pluginfunktionalität bereitstellt. Sie bietet Funktionalitäten zum erzeugen neuer parametrisierter Instanzen der Funktionalitätsklasse. Eine Liste der geforderten Konfigurationsparameter kann abgefragt werden.

Parameter:
\begin{itemize}\itemsep -10pt
\item \texttt{<T>} Der Typ der Klasse, welche die Pluginfunktionalität bereit stellt. 
\end{itemize}

Methoden:
\begin{itemize}\itemsep -10pt
\item \texttt{PluginRenderer getRenderer()}
\item[] Gibt eine \texttt{PluginRenderer}-Instanz zum Erzeugen eines Konfigurations-Steuerelements für das Plugin zurück
\item[] Returns: eine Instanz eines \texttt{PluginRenderer}s
\item \texttt{abstract List<Parameter> getParameters()}
\item[] Gibt eine Liste der Konfigurationsparameter des Plugins zurück
\item[] Returns: eine Liste der Konfigurationsparameter
\item \texttt{abstract T getNewInstance(List<double> params)}
\item[] Erzeugt eine neue Instanz der Klasse, die die Pluginfunktionalität bereitstellt und parametrisiert sie mit den übergebenen Werten der Konfigurationsparametern
\item[] \texttt{params}: eine Liste mit den Werten der Konfigurationsparameter
\item[] Returns: eine neue Instanz der Klasse, die die Pluginfunktionalität zur Verfügung stellt
\end{itemize}

\section{Glossar}
\textbf{Elementare Konfiguration:}
Eine Konfiguration, in der Multikonfiguration deaktiviert ist. Mit den \enquote{einer Konfiguration zugehörigen elementaren Konfigurationen} wird im Falle einer Multikonfiguration die Menge aller elementaren Konfigurationen bezeichnet, in denen der Multikonfigurationsparameter die festgelegte Wertemenge durchläuft. Im Falle einer elementare Konfiguration ist wieder die Konfiguration selbst gemeint.

\end{document}
